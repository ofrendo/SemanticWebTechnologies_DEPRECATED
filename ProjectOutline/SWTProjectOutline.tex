% do not change these two lines (this is a hard requirement
% there is one exception: you might replace oneside by twoside in case you deliver 
% the printed version in the accordant format
\documentclass[11pt,titlepage,oneside,openany]{article}
\usepackage{times}


\usepackage{graphicx}
\usepackage{latexsym}
\usepackage{amsmath}
\usepackage{amssymb}

\usepackage{ntheorem}

% \usepackage{paralist}
\usepackage{tabularx}

% this packaes are useful for nice algorithms
\usepackage{algorithm}
\usepackage{algorithmic}

% well, when your work is concerned with definitions, proposition and so on, we suggest this
% feel free to add Corrolary, Theorem or whatever you need
\newtheorem{definition}{Definition}
\newtheorem{proposition}{Proposition}


% its always useful to have some shortcuts (some are specific for algorithms
% if you do not like your formating you can change it here (instead of scanning through the whole text)
\renewcommand{\algorithmiccomment}[1]{\ensuremath{\rhd} \textit{#1}}
\def\MYCALL#1#2{{\small\textsc{#1}}(\textup{#2})}
\def\MYSET#1{\scshape{#1}}
\def\MYAND{\textbf{ and }}
\def\MYOR{\textbf{ or }}
\def\MYNOT{\textbf{ not }}
\def\MYTHROW{\textbf{ throw }}
\def\MYBREAK{\textbf{break }}
\def\MYEXCEPT#1{\scshape{#1}}
\def\MYTO{\textbf{ to }}
\def\MYNIL{\textsc{Nil}}
\def\MYUNKNOWN{ unknown }
% simple stuff (not all of this is used in this examples thesis
\def\INT{{\mathcal I}} % interpretation
\def\ONT{{\mathcal O}} % ontology
\def\SEM{{\mathcal S}} % alignment semantic
\def\ALI{{\mathcal A}} % alignment
\def\USE{{\mathcal U}} % set of unsatisfiable entities
\def\CON{{\mathcal C}} % conflict set
\def\DIA{\Delta} % diagnosis
% mups and mips
\def\MUP{{\mathcal M}} % ontology
\def\MIP{{\mathcal M}} % ontology
% distributed and local entities
\newcommand{\cc}[2]{\mathit{#1}\hspace{-1pt} \# \hspace{-1pt} \mathit{#2}}
\newcommand{\cx}[1]{\mathit{#1}}
% complex stuff
\def\MER#1#2#3#4{#1 \cup_{#3}^{#2} #4} % merged ontology
\def\MUPALL#1#2#3#4#5{\textit{MUPS}_{#1}\left(#2, #3, #4, #5\right)} % the set of all mups for some concept
\def\MIPALL#1#2{\textit{MIPS}_{#1}\left(#2\right)} % the set of all mips





\begin{document}

\pagenumbering{roman}
% lets go for the title page, something like this should be okay
\begin{titlepage}
	\vspace*{2cm}
  \begin{center}
   {\Large IE650 Semantic Web Technologies\\}
   \vspace{2cm} 
   {Project Outline\\}
   \vspace{2cm}
   {presented by\\
    Oliver Frendo (1510432) \\
    Sascha Ulbrich () \\
   }
   \vspace{1cm} 
   {submitted to the\\
    Data and Web Science Group\\
    Prof.\ Dr.\ Paulheim\\
    University of Mannheim\\} \vspace{2cm}
   {October 2016}
  \end{center}
\end{titlepage} 

% no lets make some add some table of contents
%\tableofcontents
%\newpage

%\listofalgorithms

%\listoffigures

%\listoftables

% evntuelly you might add something like this
% \listtheorems{definition}
% \listtheorems{proposition}

\newpage


% okay, start new numbering ... here is where it really starts
\pagenumbering{arabic}

\section{Goal of the application}
� What is the goal of the application you are going to build?
- Allow the user to automatically retrieve additional information about entities
in a given text
- As an example: A section of text containing the entities ``SAP\_SE'' and
``Hasso Plattner'' is marked and sent to the server to processing. As a result
the user is shown information gathered from the semantic web about these two
entities. 

\section{Example results}
� What are the example results you expect?


\section{Datasets}
� What datasets are you planning to use?
- DBpedia \footnote{http://dbpedia.org}
- Own dataset to simplify querying with sameAs: 
-- e.g. dbp:homepage = dbp:website = foaf:homepage

- To avoid the complexity of Web Data Integration, we treat the problem of
data fusion as out of scope. If multiple values are retrieved for a property we
use simples rules to choose which are shown to the user:
-- Numbers: Take the maximum, e.g. number of employees it is assumed that the
company will keep on growing
-- Strings: DBpedia is treated as having a higher provenance and will be used
as the default. If no values are found in DBpedia, Yago is used 



\section{Technical architecture}
� What techniques are you going to use?
- architecture picture
- Chrome extension (Chrome storage API to enable user customizing which
properties he wants to see)
- Java server (e.g. on Heroku)
-- answers to HTTP requests of Chrome extension
--- 1) GET /getTextInformation (for a given text AND given property set)
--- 2) GET /getAvailableProperties (for Person OR Location OR Organization)
-- runs CoreNLP to do Named Entity Recognition
-- Generates SPARQL queries to find information on entities and properties
-- Sends SPARQL queries to DBpedia API/Yago API



� How do you plan to evaluate your results?















\end{document}
